\section{Motivation and Background}

\noindent Nowadays the amount of information and resources available on the Web is huge and ever increasing, so that it has exceeded our ability for locating and accessing the resources we need. This way, increasingly sophisticated computational tools are required for organizing, searching and understanding those resources, beyond the traditional information indexing and retrieval approaches.

Semantic annotation, is one of the core concepts of the current proposal. It aims to make explicit for machines, the meaning (the semantics) of content and resources available in large repositories of information. This latter constitutes one of the requirements to meet to finally materializing the Semantic Web. The semantic annotation procedure is commonly supported in formal representation of knowledge, as the aforementioned ontologies, and for services, consists in associating ontological entities to the terms defining the attributes of the service in its descriptor document [8], allowing for instance, for service search engines to effectively comprehend (on a semantic level) both the services functionality as the service’s  clients requests, enabling them to accurately respond to service inquiries.

Traditionally, this semantic annotation procedure must be performed by hand by service designers and developers or in a collaborative way by service users (conceiving a sort of folksonomy of services). In both cases, the large and growing amount of services, along with the lack of knowledge regarding semantic description methods for services and the scarceness of suitable domain ontologies, has overwhelmed the human ability for performing this semantic annotation task. Additionally, the human intervention in marking up the services descriptors with ontological entities involves a very expensive process in terms of time, effort and resources. In this regard, the focus of the present approach is on leveraging current techniques taken from the fields of machine learning, information retrieval and knowledge discovery, for automating the semantic annotation of web services. The next section will deal the revision of some previous works regarding the problem being tackled herein.