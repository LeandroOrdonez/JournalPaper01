\section{Related Work}
\label{sec:related_work}

\noindent At the end of the last section we reviewed some works regarding REST APIs description. This section continues with that review but focuses on works addressing semantic annotation of Web services and resources in general.

In \cite{Loutas:2010} Loutas et al. explore alternative approaches for semantic annotation of available services and resources in the Web. In this work the author conceive information constructs derived from collaborative tagging systems (also known as folksonomies) as specifications of shared knowledge, which may be suitable for associate semantic annotations to service interfaces. However, as stated by Martinez-Cruz et al. in \cite{MartinezCruz:2010}, the wide-open nature of folksonomies involves some shortcomings in terms of organizing, searching and retrieval of resources based on tags, due to its lack of formal semantics. That way, the authors of \cite{MartinezCruz:2010} propose to build an \textit{ontology-based semantic layer} on top of these collaborative tagging systems to formalize the knowledge gathered within them. 

In other related work by Loutas et al. \cite{Loutas:2012}, they introduce a proposal for a search engine for web services, which operates on service descriptors specified in SA-REST or SAWSDL. In order to deal with heterogeneous service description formats, the authors define a comprehensive \textit{Reference Service} model to which all the crawled services are mapped. The main shortcoming of this approach lies on the fact that it operates on semantic description formats whose adoption is rather limited.

The approach outlined in \cite{Azmeh:2010} by Azmeh et al. pose the use of techniques of machine learning such as Formal Concept Analysis (FCA) and Relational Concept Analysis (RCA), for extracting and representing the technical information contained in service descriptors as conceptual hierarchies. However, the approach introduced in this work is not that exhaustive when it comes to identify similarity between services, since it relies on syntactic comparison of the terms comprising the service interfaces (i.e. WSDLs). 

The approach we propose contributes towards automating the process of semantic annotation of web services descriptors, by combining techniques of text mining and unsupervised machine learning (i.e. Latent Dirichlet Allocation–LDA) for enabling  automatic and incremental generation of a formal model of knowledge from existing service documentation sources. Such model is meant to be used in annotating and categorizing services, through a platform that implements the above techniques. 

Next section will address the description of our proposal, by outlining each one of the three processes stated at the end of section \ref{sec:motivation_background}.
