\section{Related Work}
\label{sec:related_work}

\noindent At the end of the last section we reviewed some works regarding description of REST APIs. This section continues with that review but focuses on works addressing semantic annotation of Web services and resources in general.

In \cite{Loutas:2010} Loutas et al. explore alternative approaches for semantic annotation of available services and resources in the Web. Such an approach consists of recognizing the information constructs from collaborative tagging systems (also known as folksonomies) as specifications of shared knowledge, which can be suitable for associate semantic annotations to service interfaces, dispensing with the use of ontologies. The main goal of these proposals, however, is to assist the process of semantic enrichment, still requiring human intervention (developers, users, providers, etcetera) for fulfilling the complete process. 

The authors of [12] and [13] address two works regarding to semantic annotation of folksonomies, for various kinds of online available resources. In contrast to aforementioned works, the proposals of Angeletou in [12] and the one described by Siorpaes in [13] argue that it is required to formalize the knowledge generated within folksonomies, by using ontologies, in order to overcome their limitations in terms of organizing, searching and retrieving resources based on tags.

The work of Angeletou differs from the current proposal, as long as the former is focused on an image folksonomy. In turn, the project addressed in [13], although it takes into account the services as part of its working resources, its scope is limited to promote collaborative tagging thereof.

The approaches outlined in [14, 15, 16, 17] pose the use of techniques of machine learning such as Formal Concept Analysis (FCA) and most recently Relational Concept Analysis (RCA), for extracting and representing the knowledge covered by documental corpus, as conceptual hierarchies or taxonomies. This way, the approaches described in these works are suitable for composing formal models of knowledge, such as core ontologies, avoiding the intervention of domain experts. However, none of the aforesaid proposals had considered the automation of such a process.

From observations made on related proposals, the present work aims to automate the process of semantic annotation of web services descriptors, through an approach that combines techniques of text mining, unsupervised machine learning (Latent Dirichlet Allocation–LDA) for enabling  automatic and incremental generation of formal models of knowledge from service descriptors. Such models are meant to be used in annotating and categorizing services, through a platform that implements the above techniques. 

Next section will address the description of our proposal, by outlining the architecture of the platform for automatic semantic annotation of service descriptors.
