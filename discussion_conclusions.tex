\section{Discussion and Conclusions}
\label{sec:discussion_conclusions}

\noindent The literature review conducted in developing our research, revealed the benefits of adopting the REST architectural style for building scalable distributed hypermedia systems, like the Web itself. Thanks to the HATEOAS principle of the REST uniform interface, it is possible for automated agents to carry out processes of discovery and composition of services by leveraging the hypermedia controls included in resources representations-- a dynamic that emulates the way humans browse the web---. One of the goals of the Semantic Web consists precisely in enabling computers to understand the underlying meaning of resources, so that they are able to autonomously perform service discovery, invocation and composition. This way, implementing distributed systems that adopt the REST architectural style, turns out to be a practice aligned with the realization of the Semantic Web. However, recent studies has shown that despite the apparent spread of REST services/APIs on the Web, the HATEOAS principle is in most of the cases neglected or misinterpreted.

Bearing this limitation in mind, the research work documented in this paper, proposes an approach that leverages on existing Web services and their associated documentation sources for generating a knowledge representation which captures the semantics that defines them, in a machine readable format. Such a knowledge representation allows to arrange the services into an structure that reveals semantic relationships among them.

Given a characterization based on the Richardson's maturity model for web services \cite{Richardson:2008}, in this work three types of services are discriminated: SOAP, XML-RPC and REST. For each kind of services we analyze their associated documentation sources to define which information is relevant for setting up the above knowledge representation, as well as the procedure for extracting such information.

The knowledge representation is derived by applying an online variant of the LDA topic model on the information extracted from different Web services documentation sources. This model allows deducing a set of categories that cluster semantically-related service operations and resources. The derived structure of categories is specified by using a standard format based on the RDF data model, and stored into an RDF triplestore.

Finally, the proposed techniques supported the implementation of a prototype for categorizing a set of operations included in SOAP services available online. %The results that the developed prototype delivers are promising, however an experimental evaluation that allows to objectively estimate the precision of the generated categorization is still required. This way two main steps have to be done: first, extending the implemented prototype for supporting REST and XML-RPC service documentation analysis, and lastly, perform the experimental assessment of the approach.%